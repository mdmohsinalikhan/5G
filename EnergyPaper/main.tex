%%%%%%%%%%%%%%%%%%%%%%% file template.tex %%%%%%%%%%%%%%%%%%%%%%%%%
%
% This is a general template file for the LaTeX package SVJour3
% for Springer journals.          Springer Heidelberg 2010/09/16
%
% Copy it to a new file with a new name and use it as the basis
% for your article. Delete % signs as needed.
%
% This template includes a few options for different layouts and
% content for various journals. Please consult a previous issue of
% your journal as needed.
%
%%%%%%%%%%%%%%%%%%%%%%%%%%%%%%%%%%%%%%%%%%%%%%%%%%%%%%%%%%%%%%%%%%%
%
% First comes an example EPS file -- just ignore it and
% proceed on the \documentclass line
% your LaTeX will extract the file if required
\begin{filecontents*}{example.eps}
%!PS-Adobe-3.0 EPSF-3.0
%%BoundingBox: 19 19 221 221
%%CreationDate: Mon Sep 29 1997
%%Creator: programmed by hand (JK)
%%EndComments
gsave
newpath
  20 20 moveto
  20 220 lineto
  220 220 lineto
  220 20 lineto
closepath
2 setlinewidth
gsave
  .4 setgray fill
grestore
stroke
grestore
\end{filecontents*}
%
\RequirePackage{fix-cm}
%
%\documentclass{svjour3}                     % onecolumn (standard format)
%\documentclass[smallcondensed]{svjour3}     % onecolumn (ditto)
\documentclass[twocolumn]{svjour3}       % onecolumn (second format)
%\documentclass[twocolumn]{svjour3}          % twocolumn
%
\smartqed  % flush right qed marks, e.g. at end of proof
%
\usepackage{graphicx}
%
% \usepackage{mathptmx}      % use Times fonts if available on your TeX system
%
% insert here the call for the packages your document requires
%\usepackage{latexsym}
% etc.
%
% please place your own definitions here and don't use \def but
% \newcommand{}{}
%
% Insert the name of "your journal" with
% \journalname{myjournal}
%
\begin{document}

\title{AES and SNOW 3G are feasible choices for a 5G phone from energy and throughput perspective %\thanks{Grants or other notes
%about the article that should go on the front page should be
%placed here. General acknowledgments should be placed at the end of the article.}
}
%\subtitle{Do you have a subtitle?\\ If so, write it here}

%\titlerunning{Short form of title}        % if too long for running head

\author{Mohsin Khan         \and
        Valtteri Niemi %etc.
}

%\authorrunning{Short form of author list} % if too long for running head

\institute{Mohsin Khan \at
              University of Helsinki \\
              %Tel.: +123-45-678910\\
              %Fax: +123-45-678910\\
              \email{mohsin.khan@helsinki.fi}           %  \\
%             \emph{Present address:} of F. Author  %  if needed
           \and
           Valtteri Niemi \at
              University of Helsinki \\
              %Tel.: +123-45-678910\\
              %Fax: +123-45-678910\\
              \email{valtteri.niemi@helsinki.fi}  
}

\date{Received: date / Accepted: date}
% The correct dates will be entered by the editor


\maketitle

\begin{abstract}
The aspirations for a 5th generation (5G) mobile network are high. It has a vision of unprecedented data-rate and extremely pervasive connectivity. To cater such aspirations in a mobile phone, many existing efficiency aspects of a mobile phone need to be reviewed. We look into the matter of required energy to encrypt and decrypt the huge amount of traffic that will leave from and enter into a 5G enabled mobile phone. In this paper, we present an account of the power consumption details of the efficient hardware implementations of AES and SNOW 3G. We also present an account of the power consumption details of LTE protocol stack on some cutting edge hardware platforms. Based on the aforementioned two accounts, we argue that the energy requirement for the current encryption systems AES and SNOW 3G will not impact the battery-life of a 5G enabled mobile phone by any significant proportion.
\keywords{5G \and Cryptosystem \and ASIC}
% \PACS{PACS code1 \and PACS code2 \and more}
% \subclass{MSC code1 \and MSC code2 \and more}
\end{abstract}

\section{Introduction}
\label{intro} To facilitate our discussion, we need to know what are the data that will be encrypted and decrypted in a 5G phone. We also need to know where and how many times the encryption and decryption will take place across the protocol stack on the phone. But 5G is not yet a reality and we do not have exact answers to these questions. So, we assume things, that will be true for a 5G network and argue on the basis of those assumptions. We turn to the LTE network to make the assumptions. In an LTE phone, the data that leave and enter the phone can be broadly classified into three categories. The first one are the control signals in between the phone and the core network. The second one are the control signals in between the phone and the radio network. And the third one are the user data which the user sends from and receive at it's application layer. Both of the first two categories are privacy and integrity protected. For the third category, only the privacy is protected. Also note that, from the volume point of view, the major share of data belong to the third category. And comparing to the the third category, the cryptographic computational need required for the data of first and second categories is negligible. And the user data in an LTE phone is only once encrypted and decrypted across the protocol stack in PDCP layer. In an LTE phone this encryption is done by an application specific integrated circuit (ASIC). For a 5G phone, we assume that the user data will remain as the major share of the total data leaving and entering the phone. And the cryptographic computational need for the total volume of control signals will be negligible in comparison with that of the user data. And that the user data will only once be encrypted and decrypted somewhere across the protocol stack. From hardware point of view it will still be in an ASIC. In order to have a pessimistic estimation, we assume that integrity protection of user data will be introduced in 5G. Based on these assumptions, we will look into the cryptographic energy requirements and also the total energy requirements across the whole protocol stack of an LTE phone. Then we will scale up the data-rate from 100 Mbps to 1 Gbps and see how much extra pressure it puts on the battery of the phone in comparison with other energy hungry aspects of the phone like display and radio signalling.


According to \cite{EEA1_EEA2}, there are two encryption schemes in the 4th generation cellular network (LTE) developed by 3GPP. One is EEA1 in which the stream cipher SNOW 3G is used. The other is EEA2 in which the block cipher AES is used. According to the LTE architecture, there are three different connections  are two encryption systems used to encrypt the user data traffic in 3GPP-defined cellular networks across the radio layer. As the data rate of the cellular networks has increased steadily throughout the history of the networks, researchers have focused on implementing these cryptosystems both in hardware and software to achieve the required throughput. Looking up in the existing literature, it has been found that there exists implementations of these two cryptosystems that can achieve the required throughput even for a 5G network, that is at least 1 Gbps. However, there is no concrete account available of the power consumption of these implementations that enables the readers to estimate the energy share of the task of encryption across the entire protocol stack. We have studied, collected and rendered the relevant information available in the literature into this single article in an easily comprehensible and comparable manner and make a case that the energy share of data encryption is too low to think of any alternative lightweight encryption for 5G enabled mobile phone.


\section{Encryption is time consuming}
\label{sec:encryption_is_time_consuming}
It has been identified that encryption is the most time consuming process across the downlink (DL) of layer 2 (L2) in a mobile phone when traditional hardware acceleration concepts have been used. In a study \cite{IIS_Ruhr_2009} published in 2009, the authors in their experiment have found that 68 percent of the execution time spent in the L2 DL in LTE phone was consumed by deciphering when AES has been used considering the state-of-the-art mobile platform of the time. They also showed that instead of traditional hardware acceleration concepts more sophisticated hardware accelerators for the L2 are needed to supply enough computational power required in LTE and next generation mobile devices. However, they do not give any account of the case where SNOW 3G has been used. Nevertheless, other studies suggest that AES and SNOW 3G has very similar kind of throughput properties in the state of the art implementations. In a study in \cite{IIS_Ruhr_2010} conducted in 2010 on the L2 DL layer (MAC,PDCP,RLC) layer, it has has been shown that by an sDMA, the authors did not mention anything about the achieved throughput but said that, it is enough for an LTE terminal. However, to achive this required throughput, the implementation consumed 9.5 mW of power whereas AES and SNOW 3G each required .5 and .57 mW of power respectively. Which means the encryption/decryption consumes around 5 percent of the power budget of L2 DL. (see in figure 6)AES

Whereas, a very recent study done in 2014, presented in \cite{KTH_2014}, conducted on the UDP/IP layer, shows that (in Table II) an ASIC implementation consumes $14.62$ nano Joule of energy for a Kilobyte data in this layer. Which means it takes $(14.62/8)*1000 = 1827$ micro Joule of energy for 1 Giga bit data while providing throughtput of 2.24 Gbps.


\section{AES}
\label{sec:aes}

Since the adoption of Rijndael as AES by NIST, there have been number of hardware implementations of AES to achieve efficiency and high throughput. The below table gives a picture \newline


%\begin{tabular}{ |p{1cm}|p{.7cm}|p{.5cm}|p{1.2cm}|p{1cm}|p{1cm}|p{1cm}|p{1cm}|p{2cm}|p{1cm}|}
%\hline
%\multicolumn{10}{|c|}{AES Implementations} \\
%\hline
%Year &Tech &Ref &TP (Gbps) &Gates (K) & Clock Speed (MHz)&Power (mW) & Scal-able (Y/N) & Throughput per Kilo Gates in Gpbs & Energy per Gbit in $\mu$Joule \\ %\hline
%2001 & $.11 \mu$ & \cite{IBM_Japan_2001} &$2.6$ &$21.3$  & 0& - & Y & 0.122 & - \\ \hline
%2001 & $.11 \mu$ & \cite{IBM_Japan_2001} &$.311$ &$5.4$  & 0& - & Y & 0.0576 & - \\ \hline
%2001 & - & \cite{IBM_India_IIT_2001} &$.24$ &$4$  & 0& - & Y & 0.06 & - \\ \hline
%2006 & $.18 \mu$ & \cite{Taiwan_2006} &$.570$ & - & 48&20.34 &Y & - & 35684 \\ \hline
%2006 & $.35 \mu$ & \cite{Taiwan_2006} &$.569$ & - & 48&192.5 &Y & - & 338312 \\ \hline
%2007 & $.18 \mu$ & \cite{IIT_Kharagpur_2007} &$.384$ &$21$ & 120&- &Y & 0.018 & - \\ \hline
%2009 & $.18 \mu$ & \cite{IME_China_Tsinghua_Univerisity_2009} &$1.16$ &$19.47$ & - & - &Y & 0.056 & - \\ \hline
%2009 & $.09 \mu$ & \cite{Ruhr_2009} &$1.86$ &$15.25$ & 450&.78 &Y & 0.015 & 419 \\ \hline
%2011 & $ - $ & \cite{Ruhr_2011} &$.114$ &$ - $ & 300&.02 &Y & -  & 186.18 \\ \hline
%2012 & $.18 \mu$ & \cite{Pune_2012} &$1.6$ &$58.445$ & 125&22.85 &Y & - & 14281 \\ \hline
%\end{tabular}



So far the best power figure is found in \cite{Ruhr_2009}. The implementation is scalable. Interestingly the power figure doesn't increase linearly with the number of AES engines used. Let us assume that the required data rate in 5G is 1 Gbps. So, the timing requirement is: 7.45 nano seconds per byte. 


According to figure 9 in \cite{Ruhr_2011}, it will take roughly $.02*5=.1 mW$ for 1Gbps throughput. Because SAME has throughput of 114Mbps. And it achieves $5.5Mbps/ \mu J$. Which means it spends $114/5.5 = 20.72\mu J = .02 mW$. Now, by scaling up by 10 times, it will spend $.02*10=.2mW$ to achieve 1.14Gbps throughput


Comparing the best figure from the above table with \cite{KTH_2014}, we see that ciphering takes $186/1827=.1$ or 10 percent energy of data communication on a ASIC packet processor.


In the study in \cite{IIS_Ruhr_2010}, it was found that encryption was taking 5 or 6 percent of the power budget. However, as because other aspects of the protocol stack could be made more efficient, the power budget of encryption/decryption has increased.

So, with the current best implementation of the encryptor/decryptor hardware engine, in near future the power budget will only increase. However, now we need to check if 10 percent of the power budget for encryption will be good enough for a 5G phone or not.

%\begin{acknowledgements}
%If you'd like to thank anyone, place your comments here
%and remove the percent signs.
%\end{acknowledgements}

% BibTeX users please use one of
%\bibliographystyle{spbasic}      % basic style, author-year citations
%\bibliographystyle{spmpsci}      % mathematics and physical sciences
%\bibliographystyle{spphys}       % APS-like style for physics
%\bibliography{}   % name your BibTeX data base

% Non-BibTeX users please use
\begin{thebibliography}{}
%
% and use \bibitem to create references. Consult the Instructions
% for authors for reference list style.
%

\bibitem{EEA1_EEA2}
3GPP, Article title, Journal, Volume, page numbers (year)


\bibitem{RefJ}
% Format for Journal Reference
Author, Article title, Journal, Volume, page numbers (year)

% Format for books
\bibitem{RefB}
Author, Book title, page numbers. Publisher, place (year)
% etc
\end{thebibliography}

\end{document}
% end of file template.tex
