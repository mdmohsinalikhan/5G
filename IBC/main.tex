
%%%%%%%%%%%%%%%%%%%%%%% file typeinst.tex %%%%%%%%%%%%%%%%%%%%%%%%%%%%%%
%
% This is the LaTeX source for the instructions to authors using
% the LaTeX document class SVMultln with class option 'lnicst'
% for contributions to the Lecture Notes of the Institute for
% Computer Sciences, Social-Informatics and
% Telecommunications Engineering series.
% www.springer.com/series/XXXX       Springer Heidelberg 2007/08/05
%
% It may be used as a template for your own input - copy it
% to a new file with a new name and use it as the basis for
% your article. It contains a few tweaked sections to demonstrate
% features of the package, though.
%
% If you have not much experiences with Springer LaTeX support,
% you should better use the special demonstration file "lnicst.tex"
% included in the LaTeX package for LNICST as template.
%
%%%%%%%%%%%%%%%%%%%%%%%%%%%%%%%%%%%%%%%%%%%%%%%%%%%%%%%%%%%%%%%%%%%%%%%%

\documentclass[lnicst,sechang,a4paper]{svmultln}
\usepackage{amssymb}
\setcounter{tocdepth}{3}
\usepackage{graphicx}

\usepackage{url}
\urldef{\mailsa}\path|{mohsin.khan, valtteri.niemi}@helsinki.fi|
\usepackage[pdfpagelabels,hypertexnames=false,breaklinks=true,bookmarksopen=true,bookmarksopenlevel=2]{hyperref}


%added by the author himself
\usepackage{color}
\usepackage[numbers]{natbib}
\usepackage{calc}
\usepackage{siunitx}
\DeclareSIUnit\mt{\milli\tesla} %% A method for say short cut or new unit!
\sisetup{inter-unit-product = {-}}

\begin{document}

\mainmatter  % start of an individual contribution

% first the title is needed
\title{Protection Against the IMSI Catchers Using Identity Based Crypto in 5G}

% a short form should be given in case it is too long for the running head
\titlerunning{Identity Based Crypto Against IMSI Catchers}

% the name(s) of the author(s) follow(s) next
%
% NB: Chinese authors should write their first names(s) in front of
% their surnames. This ensures that the names appear correctly in
% the running heads and the author index.
%
\author{Mohsin Khan%
%%\thanks{Please note that the LNICST Editorial assumes that all authors have used
%%the western naming convention, with given names preceding surnames. This determines
%%the structure of the names in the running heads and the author index.}%
\and Valtteri Niemi}  %
%\authorrunning{Lecture Notes of ICST: Authors' Instructions}
% (feature abused for this document to repeat the title also on left hand pages)

% the affiliations are given next
\institute{University of Helsinki, Department of Computer Science,\\
P.O. Box 68 (Gustaf H\"allstr\"amin katu 2b)\\
FI-00014 University of Helsinki\\
Finland\\
\mailsa
%\\
%\url{http://www.springer.com/series/7911}
}

%
% NB: a more complex sample for affiliations and the mapping to the
% corresponding authors can be found in the file "lnicst.dem",
% that is contained in the LNICST LaTeX support package.
%

\toctitle{AES and SNOW 3G are Feasible Choices}
\tocauthor{Authors' Instructions}
\maketitle


\begin{abstract}
The aspirations for the next generation mobile network (5G) are high. It has a vision of improving security and privacy over the existing LTE network. Identity privacy of a user has been a historical concern with all the previous generation mobile networks, namely GSM, UMTS, and LTE. While little improvements have been achieved in securing the privacy of long term identity of a user, the so called IMSI catchers are still in existence even in the LTE and advanced LTE networks. This report looks into this privacy problem and presents different techniques of using public key cryptography to tackle it. One special case of public key crytography is identity based crypto. A rigorous comparison among the pros and cons of these different techniques show that identity based crypto is a potential solution for securing the identity privacy of a user in the 5G network.
\end{abstract}


\section{Introduction}
\label{intro} NGMN Alliance has pointed out identity privacy of a user as a requirement of the 5G network under the requirement category of enhanced services \cite{NGMN_white_paper}. In 3GPP TR 33.899 \cite{TR33899}, subscribers' privacy is captured as one of the high level security requirements of the 5G network. However, in the context of diversified devices and complex business and service model of 5G, it is important to define who is a subscriber and what subscriber privacy means. According to 3GPP TR 21.905 \cite{TR21905} a subscriber is an entity (associated with one or more users) that is engaged in a Subscription with a service provider. A Subscription describes the commercial relationship between the subscriber and the service provider, cf. 3GPP TR 21.905 [1]. Subscription Identifier: The identifier that uniquely identifies a subscription in the 3GPP system. The identifier is used to access networks based on 3GPP specifications. Privacy refers to the right to the protection to any information that (a) can be used to identify a subscription to whom such information relates, or (b) is or might be directly or indirectly linked to a subscription. Privacy requirements: Set of requirements to take into account when a 3GPP node is processing personally identifiable information (PII). 

\textcolor{red}{What are the privacy requirements in the context of subscription privacy?}

\section{Public key cryptography against IMSI catchers}
Here we use public key cryptography which may or may not be based on identity based crypto to secure the privacy of the long term identity of a mobile phone user called IMSI (International mobile subscriber identity). We discuss different techniques of using the public key cryptography:

\begin{enumerate}
\item Identity based crypto based on the identity of SN where the HN is the key generator
\item HN assigned public private key pair for each SN
\item HN owned public private key pair
\end{enumerate}

In the consequent sections we describe the aforementioned techniques in further detail.

\section{Based on Identity of Serving Network} In this technique the HN has a public and private key pair. Every phone knows the public key of the HN. Whenever a SN asks the phone to provide its IMSI, the phone computes the public key of the SN using the public key of the HN. Then the phone encrypts the IMSI with the computed public key of the SN and sends it to the SN along with the HN identity. The SN obtains (possibly already have obtained) its private key from the mentioned HN. Using this private key, the SN can decrypt IMSI. Figure \ref{fig:IBC} represents the high level protocol.

\subsection{Concerns and Solutions}
\begin{enumerate}
\item How to provision, revoke and re-provision the public key of HN in the phone?
\item How to black list a SN?
\end{enumerate}

\subsection{Based on HN generated public private key pair for every SN}

\section{Conclusion}
\label{sec:conclusion}

\section{Acknowledgement}
\label{sec:acknowledgement}



\begin{thebibliography}{4}

\bibitem{NGMN_white_paper} NGMN 5G White Paper V1.0 [cited Jan, 2017]. Available at: https://www.ngmn.org/uploads/media/NGMN\_5G\_White\_Paper\_V1\_0.pdf


\bibitem{TR33899} 3GPP TR 33.899 V0.6.0 [cited Jan, 2017]. Available at: https://portal.3gpp.org/desktopmodules/Specifications/SpecificationDetails.aspx?specificationId=3045
\end{thebibliography}

\end{document}
